% Options for packages loaded elsewhere
\PassOptionsToPackage{unicode}{hyperref}
\PassOptionsToPackage{hyphens}{url}
%
\documentclass[
]{article}
\usepackage{amsmath,amssymb}
\usepackage{lmodern}
\usepackage{ifxetex,ifluatex}
\ifnum 0\ifxetex 1\fi\ifluatex 1\fi=0 % if pdftex
  \usepackage[T1]{fontenc}
  \usepackage[utf8]{inputenc}
  \usepackage{textcomp} % provide euro and other symbols
\else % if luatex or xetex
  \usepackage{unicode-math}
  \defaultfontfeatures{Scale=MatchLowercase}
  \defaultfontfeatures[\rmfamily]{Ligatures=TeX,Scale=1}
\fi
% Use upquote if available, for straight quotes in verbatim environments
\IfFileExists{upquote.sty}{\usepackage{upquote}}{}
\IfFileExists{microtype.sty}{% use microtype if available
  \usepackage[]{microtype}
  \UseMicrotypeSet[protrusion]{basicmath} % disable protrusion for tt fonts
}{}
\makeatletter
\@ifundefined{KOMAClassName}{% if non-KOMA class
  \IfFileExists{parskip.sty}{%
    \usepackage{parskip}
  }{% else
    \setlength{\parindent}{0pt}
    \setlength{\parskip}{6pt plus 2pt minus 1pt}}
}{% if KOMA class
  \KOMAoptions{parskip=half}}
\makeatother
\usepackage{xcolor}
\IfFileExists{xurl.sty}{\usepackage{xurl}}{} % add URL line breaks if available
\IfFileExists{bookmark.sty}{\usepackage{bookmark}}{\usepackage{hyperref}}
\hypersetup{
  pdftitle={Data exploration},
  pdfauthor={Desiree Smith},
  hidelinks,
  pdfcreator={LaTeX via pandoc}}
\urlstyle{same} % disable monospaced font for URLs
\usepackage[margin=1in]{geometry}
\usepackage{color}
\usepackage{fancyvrb}
\newcommand{\VerbBar}{|}
\newcommand{\VERB}{\Verb[commandchars=\\\{\}]}
\DefineVerbatimEnvironment{Highlighting}{Verbatim}{commandchars=\\\{\}}
% Add ',fontsize=\small' for more characters per line
\usepackage{framed}
\definecolor{shadecolor}{RGB}{248,248,248}
\newenvironment{Shaded}{\begin{snugshade}}{\end{snugshade}}
\newcommand{\AlertTok}[1]{\textcolor[rgb]{0.94,0.16,0.16}{#1}}
\newcommand{\AnnotationTok}[1]{\textcolor[rgb]{0.56,0.35,0.01}{\textbf{\textit{#1}}}}
\newcommand{\AttributeTok}[1]{\textcolor[rgb]{0.77,0.63,0.00}{#1}}
\newcommand{\BaseNTok}[1]{\textcolor[rgb]{0.00,0.00,0.81}{#1}}
\newcommand{\BuiltInTok}[1]{#1}
\newcommand{\CharTok}[1]{\textcolor[rgb]{0.31,0.60,0.02}{#1}}
\newcommand{\CommentTok}[1]{\textcolor[rgb]{0.56,0.35,0.01}{\textit{#1}}}
\newcommand{\CommentVarTok}[1]{\textcolor[rgb]{0.56,0.35,0.01}{\textbf{\textit{#1}}}}
\newcommand{\ConstantTok}[1]{\textcolor[rgb]{0.00,0.00,0.00}{#1}}
\newcommand{\ControlFlowTok}[1]{\textcolor[rgb]{0.13,0.29,0.53}{\textbf{#1}}}
\newcommand{\DataTypeTok}[1]{\textcolor[rgb]{0.13,0.29,0.53}{#1}}
\newcommand{\DecValTok}[1]{\textcolor[rgb]{0.00,0.00,0.81}{#1}}
\newcommand{\DocumentationTok}[1]{\textcolor[rgb]{0.56,0.35,0.01}{\textbf{\textit{#1}}}}
\newcommand{\ErrorTok}[1]{\textcolor[rgb]{0.64,0.00,0.00}{\textbf{#1}}}
\newcommand{\ExtensionTok}[1]{#1}
\newcommand{\FloatTok}[1]{\textcolor[rgb]{0.00,0.00,0.81}{#1}}
\newcommand{\FunctionTok}[1]{\textcolor[rgb]{0.00,0.00,0.00}{#1}}
\newcommand{\ImportTok}[1]{#1}
\newcommand{\InformationTok}[1]{\textcolor[rgb]{0.56,0.35,0.01}{\textbf{\textit{#1}}}}
\newcommand{\KeywordTok}[1]{\textcolor[rgb]{0.13,0.29,0.53}{\textbf{#1}}}
\newcommand{\NormalTok}[1]{#1}
\newcommand{\OperatorTok}[1]{\textcolor[rgb]{0.81,0.36,0.00}{\textbf{#1}}}
\newcommand{\OtherTok}[1]{\textcolor[rgb]{0.56,0.35,0.01}{#1}}
\newcommand{\PreprocessorTok}[1]{\textcolor[rgb]{0.56,0.35,0.01}{\textit{#1}}}
\newcommand{\RegionMarkerTok}[1]{#1}
\newcommand{\SpecialCharTok}[1]{\textcolor[rgb]{0.00,0.00,0.00}{#1}}
\newcommand{\SpecialStringTok}[1]{\textcolor[rgb]{0.31,0.60,0.02}{#1}}
\newcommand{\StringTok}[1]{\textcolor[rgb]{0.31,0.60,0.02}{#1}}
\newcommand{\VariableTok}[1]{\textcolor[rgb]{0.00,0.00,0.00}{#1}}
\newcommand{\VerbatimStringTok}[1]{\textcolor[rgb]{0.31,0.60,0.02}{#1}}
\newcommand{\WarningTok}[1]{\textcolor[rgb]{0.56,0.35,0.01}{\textbf{\textit{#1}}}}
\usepackage{graphicx}
\makeatletter
\def\maxwidth{\ifdim\Gin@nat@width>\linewidth\linewidth\else\Gin@nat@width\fi}
\def\maxheight{\ifdim\Gin@nat@height>\textheight\textheight\else\Gin@nat@height\fi}
\makeatother
% Scale images if necessary, so that they will not overflow the page
% margins by default, and it is still possible to overwrite the defaults
% using explicit options in \includegraphics[width, height, ...]{}
\setkeys{Gin}{width=\maxwidth,height=\maxheight,keepaspectratio}
% Set default figure placement to htbp
\makeatletter
\def\fps@figure{htbp}
\makeatother
\setlength{\emergencystretch}{3em} % prevent overfull lines
\providecommand{\tightlist}{%
  \setlength{\itemsep}{0pt}\setlength{\parskip}{0pt}}
\setcounter{secnumdepth}{-\maxdimen} % remove section numbering
\ifluatex
  \usepackage{selnolig}  % disable illegal ligatures
\fi

\title{Data exploration}
\usepackage{etoolbox}
\makeatletter
\providecommand{\subtitle}[1]{% add subtitle to \maketitle
  \apptocmd{\@title}{\par {\large #1 \par}}{}{}
}
\makeatother
\subtitle{Alex Fink, JT}
\author{Desiree Smith}
\date{9/17/2021}

\begin{document}
\maketitle

\begin{Shaded}
\begin{Highlighting}[]
\FunctionTok{require}\NormalTok{(}\StringTok{"here"}\NormalTok{)}
\end{Highlighting}
\end{Shaded}

\begin{verbatim}
## Loading required package: here
\end{verbatim}

\begin{verbatim}
## here() starts at /Users/desireesmith/OneDrive - University of Massachusetts/ECo 602 Lab/environmental_data
\end{verbatim}

\begin{Shaded}
\begin{Highlighting}[]
\FunctionTok{getwd}\NormalTok{()}
\end{Highlighting}
\end{Shaded}

\begin{verbatim}
## [1] "/Users/desireesmith/OneDrive - University of Massachusetts/ECo 602 Lab/environmental_data/assignments"
\end{verbatim}

\begin{Shaded}
\begin{Highlighting}[]
\FunctionTok{here}\NormalTok{()}
\end{Highlighting}
\end{Shaded}

\begin{verbatim}
## [1] "/Users/desireesmith/OneDrive - University of Massachusetts/ECo 602 Lab/environmental_data"
\end{verbatim}

\begin{Shaded}
\begin{Highlighting}[]
\NormalTok{data\_habitat }\OtherTok{=} \FunctionTok{data.frame}\NormalTok{(}\FunctionTok{read.csv}\NormalTok{(}\FunctionTok{here}\NormalTok{(}\StringTok{"data"}\NormalTok{, }\StringTok{"hab.sta.csv"}\NormalTok{)))}
\end{Highlighting}
\end{Shaded}

\begin{Shaded}
\begin{Highlighting}[]
\FunctionTok{par}\NormalTok{(}\AttributeTok{mfrow =} \FunctionTok{c}\NormalTok{ (}\DecValTok{1}\NormalTok{,}\DecValTok{3}\NormalTok{))}
\FunctionTok{hist}\NormalTok{(data\_habitat}\SpecialCharTok{$}\NormalTok{slope, }\AttributeTok{xlab =} \StringTok{"Slope"}\NormalTok{, }\AttributeTok{main =} \StringTok{"Slope of Terrain"}\NormalTok{, }\AttributeTok{col =} \StringTok{"blue"}\NormalTok{)}
\FunctionTok{hist}\NormalTok{(data\_habitat}\SpecialCharTok{$}\NormalTok{aspect, }\AttributeTok{xlab =} \StringTok{"Aspect"}\NormalTok{, }\AttributeTok{main =} \StringTok{"Aspect of Terrain"}\NormalTok{, }\AttributeTok{col =} \StringTok{"orange"}\NormalTok{)}
\FunctionTok{hist}\NormalTok{(data\_habitat}\SpecialCharTok{$}\NormalTok{elev, }\AttributeTok{xlab =} \StringTok{"Elevation"}\NormalTok{, }\AttributeTok{main =} \StringTok{"Elevation of Terrain"}\NormalTok{, }\AttributeTok{col =} \StringTok{"green"}\NormalTok{)}
\end{Highlighting}
\end{Shaded}

\includegraphics{Data_Exploration_files/figure-latex/unnamed-chunk-2-1.pdf}

\textbf{Question 2} The shape of the elevation histogram is negatively
skewed to the left. The elevations that are 500 meters and less seem to
occur more frequently. This is where a bulk of the observations were
made. The stations that were found in these areas of lower elevations
seen more birds in the area.

\textbf{Question 3} The units of slope in the data set is percent slope.

\textbf{Question 4} Unlike the elevation histogram the slope histogram
is not skewed in one direction. The shape of the slope histogram is a
bell shaped curve. The highest values are in the middle and the lower
values are on the outside. Most of the sample sites are steep but there
is still a good amount of shallow slopes.

\textbf{Question 5} An aspect is the direction that the that the
physical slope is facing.

\textbf{Question 6} Looking at the aspect histogram the values are
evenly distributed. It has a uniform look with some variation. The
sample sites tend be slightly more north-facing but there is still a
good amount of south facing. This can show that the birds may not have a
strong preference when it comes to which direction the terrain is
facing.

\textbf{Question 7}

\begin{Shaded}
\begin{Highlighting}[]
\NormalTok{line\_point\_slope }\OtherTok{=} \ControlFlowTok{function}\NormalTok{(x, x1, y1, slope)}
\NormalTok{\{}
\NormalTok{  get\_y\_intercept }\OtherTok{=} 
    \ControlFlowTok{function}\NormalTok{(x1, y1, slope) }
      \FunctionTok{return}\NormalTok{(}\SpecialCharTok{{-}}\NormalTok{(x1 }\SpecialCharTok{*}\NormalTok{ slope) }\SpecialCharTok{+}\NormalTok{ y1)}
  
\NormalTok{  linear }\OtherTok{=} 
    \ControlFlowTok{function}\NormalTok{(x, yint, slope) }
      \FunctionTok{return}\NormalTok{(yint }\SpecialCharTok{+}\NormalTok{ x }\SpecialCharTok{*}\NormalTok{ slope)}
  
  \FunctionTok{return}\NormalTok{(}\FunctionTok{linear}\NormalTok{(x, }\FunctionTok{get\_y\_intercept}\NormalTok{(x1, y1, slope), slope))}
\NormalTok{\}}

\FunctionTok{par}\NormalTok{(}\AttributeTok{mfrow =} \FunctionTok{c}\NormalTok{ (}\DecValTok{1}\NormalTok{,}\DecValTok{3}\NormalTok{))}
\FunctionTok{plot}\NormalTok{(ba.tot }\SpecialCharTok{\textasciitilde{}}\NormalTok{ slope, }\AttributeTok{data =}\NormalTok{ data\_habitat, }\AttributeTok{main =} \StringTok{"Slope"}\NormalTok{, }\AttributeTok{ylab =} \StringTok{"Basal Area"}\NormalTok{, }\AttributeTok{xlab =} \StringTok{"Slope"}\NormalTok{, }\AttributeTok{col =} \StringTok{"blue"}\NormalTok{)}
\FunctionTok{curve}\NormalTok{(}\FunctionTok{line\_point\_slope}\NormalTok{(x, }\AttributeTok{x1 =} \FloatTok{3.5}\NormalTok{, }\AttributeTok{y1 =} \FloatTok{1.25}\NormalTok{, }\AttributeTok{slope =} \FloatTok{0.3}\NormalTok{), }\AttributeTok{add =} \ConstantTok{TRUE}\NormalTok{)}
\FunctionTok{plot}\NormalTok{(ba.tot }\SpecialCharTok{\textasciitilde{}}\NormalTok{ elev, }\AttributeTok{data =}\NormalTok{ data\_habitat, }\AttributeTok{main =} \StringTok{"Aspect"}\NormalTok{, }\AttributeTok{ylab =} \StringTok{"Basal Area"}\NormalTok{, }\AttributeTok{xlab =} \StringTok{"Aspect"}\NormalTok{, }\AttributeTok{col =} \StringTok{"orange"}\NormalTok{)}
\FunctionTok{curve}\NormalTok{(}\FunctionTok{line\_point\_slope}\NormalTok{(x, }\AttributeTok{x1 =} \FloatTok{3.5}\NormalTok{, }\AttributeTok{y1 =} \FloatTok{1.25}\NormalTok{, }\AttributeTok{slope =} \FloatTok{0.05}\NormalTok{), }\AttributeTok{add =} \ConstantTok{TRUE}\NormalTok{)}
\FunctionTok{plot}\NormalTok{(ba.tot }\SpecialCharTok{\textasciitilde{}}\NormalTok{ elev, }\AttributeTok{data =}\NormalTok{ data\_habitat, }\AttributeTok{main =} \StringTok{"Elevation"}\NormalTok{, }\AttributeTok{ylab =} \StringTok{"Basal Area"}\NormalTok{, }\AttributeTok{xlab =} \StringTok{"Elevation"}\NormalTok{, }\AttributeTok{col =} \StringTok{"green"}\NormalTok{)}
\FunctionTok{curve}\NormalTok{(}\FunctionTok{line\_point\_slope}\NormalTok{(x, }\AttributeTok{x1 =} \FloatTok{3.5}\NormalTok{, }\AttributeTok{y1 =} \FloatTok{1.25}\NormalTok{, }\AttributeTok{slope =} \FloatTok{0.04}\NormalTok{), }\AttributeTok{add =} \ConstantTok{TRUE}\NormalTok{)}
\end{Highlighting}
\end{Shaded}

\includegraphics{Data_Exploration_files/figure-latex/unnamed-chunk-3-1.pdf}

\textbf{Question 8} For each terrain variable (elevation, slope,
aspect), describe the relationship you observe and your model fit. You
should consider

Is there a noticeable association? If so, is it linear? Based on a
visual assessment, is your linear model a good fit for the data, why or
why not?

When I look at the scatter plots I do not see a strong correlation in
any of the terrain variable. Looking at the slope I see not noticeable
any correlation. The Aspect and elevation model seems like there may be
a weak negative correlation but it is not strong. The data does not go
strongly in one direction. It goes straight up and slightly to the left.
The points are clumped together around the line a best fit so this can
show some strong association of the data. The line goes through a
majority of the points that were observed.

\end{document}
